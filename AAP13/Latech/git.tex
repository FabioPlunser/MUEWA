\documentclass{article}



%\usepackage[utf8]{inputenc}%\usepackage{tikz}#
\usepackage{lmodern}
\usepackage{setspace}
\usepackage{lastpage}
\usepackage{pgfplots} % Fuer Plots
\usepackage{pgfplotstable}
\usepackage{pgfkeys} 
\usepackage{filecontents}
\pgfplotsset{compat = newest}
\usepackage{tikz} % Generell gutes Packet
\usepackage{csvsimple} % Zum CSV auslesen
\usepackage{fancyhdr}
\usepackage{tcolorbox}
\usepackage{graphicx}
\usepackage{subfig}
\usepackage{pdfpages}
\usepackage{titlesec}
\usepackage{hyperref}
\usepackage{pdfpages}
\usepackage{verbatim}
\usepackage{geometry}
\geometry{left=20mm, right=20mm, bottom=20mm}

\graphicspath{{Bilder}}

\titleclass{\subsubsubsection}{straight}[\subsection]
\newcounter{subsubsubsection}[subsubsection]
\renewcommand\thesubsubsubsection{\thesubsubsection.\arabic{subsubsubsection}}
\renewcommand\theparagraph{\thesubsubsubsection.\arabic{paragraph}} % optional; useful if paragraphs are to be numbered
\titleformat{\subsubsubsection}
  {\normalfont\normalsize\bfseries}{\thesubsubsubsection}{1em}{}
\titlespacing*{\subsubsubsection}
{0pt}{3.25ex plus 1ex minus .2ex}{1.5ex plus .2ex}


\titleclass{\subsubsubsubsection}{straight}[\subsection]
\newcounter{subsubsubsubsection}[subsubsubsection]
\renewcommand\thesubsubsubsubsection{\thesubsubsubsection.\arabic{subsubsubsubsection}}
\titleformat{\subsubsubsubsection}
  {\normalfont\normalsize\bfseries}{\thesubsubsubsubsection}{1em}{}
\titlespacing*{\subsubsubsubsection}
{0pt}{3.25ex plus 1ex minus .2ex}{1.5ex plus .2ex}

\makeatletter
\renewcommand\paragraph{\@startsection{paragraph}{5}{\z@}%
  {3.25ex \@plus1ex \@minus.2ex}%p
  {-1em}%
  {\normalfont\normalsize\bfseries}}
\renewcommand\subparagraph{\@startsection{subparagraph}{6}{\parindent}%
  {3.25ex \@plus1ex \@minus .2ex}%
  {-1em}%
  {\normalfont\normalsize\bfseries}}
\def\toclevel@subsubsubsection{4}
\def\toclevel@paragraph{5}
\def\toclevel@paragraph{6}
\def\l@subsubsubsection{\@dottedtocline{4}{7em}{4em}}
\def\l@subsubsubsubsection{\@dottedtocline{5}{8em}{5em}}
\def\l@paragraph{\@dottedtocline{5}{10em}{5em}}
\def\l@subparagraph{\@dottedtocline{6}{14em}{6em}}
\makeatother

\setcounter{secnumdepth}{5}
\setcounter{tocdepth}{5}


\pagestyle{fancy}
\fancyhf{}

\title{AAP13: GIT}
\author{Fabio~Plunser\\Betreuer~:~Walter~Mueller}
\date{\today}
\rhead{4BHEL \hspace{5px}\includegraphics[scale=0.09]{Bilder/logo.png}}
\lhead{AAP13: GIT}
\rfoot{Page~\thepage ~of~\pageref{LastPage}}
\lfoot{Gruppe~H}
\renewcommand{\headrulewidth}{1pt}
\renewcommand{\footrulewidth}{1pt}


\begin{document}
\pagenumbering{gobble}

\begin{titlepage}
	\maketitle
\end{titlepage}

\pagebreak
\thispagestyle{empty}
\renewcommand\contentsname{Inhaltsverzeichnis}
\tableofcontents	
\pagenumbering{gobble}

\thispagestyle{empty}
\renewcommand\listfigurename{Abbildungsverzeichnis}
\listoffigures
\pagebreak
\pagenumbering{arabic}



%#-----------------------------------------------------------------------------------------------------------------------------------------------------
%#-----------------------------------------------------------------------------------------------------------------------------------------------------
%#-----------------------------------------------------------------------------------------------------------------------------------------------------
%#-----------------------------------------------------------------------------------------------------------------------------------------------------
%#-----------------------------------------------------------------------------------------------------------------------------------------------------
\newpage
\section{Verwendet PyCharm und klont mit dem integrierten GIT Tools ein Projekt von GitHUB}
\begin{tcolorbox}[notitle,boxrule=0pt,colback=gray!20]
\href{https://github.com/mue-htl/4bhel-aap13}{Github-Repo}\\

Last das Programm laufen, schaut Euch die Git-Historie an. 
Macht und commitet Änderungen in Eurem lokalen Repo. \\\\

Mit der Extension git-history für Visual Studio Code kann genau nachgeschaut werden wer, wann, was geändert hat.\\
Wenn viele Personen an einem Projekt arbeiten ist das wirklich unglaublich nützlich. 
\end{tcolorbox}
\begin{figure}[!h]
  \centering
  \includegraphics[scale=0.3]{Bilder/git-history}
  \caption{MUE-History on Git}
\end{figure}


\newpage
%#-----------------------------------------------------------------------------------------------------------------------------------------------------
%#-----------------------------------------------------------------------------------------------------------------------------------------------------
%#-----------------------------------------------------------------------------------------------------------------------------------------------------
%#-----------------------------------------------------------------------------------------------------------------------------------------------------
%#-----------------------------------------------------------------------------------------------------------------------------------------------------
\section{Legt einen eigenen Benutzer bei GitHub an. Legt im neuen Account auch eine Testrepository an.}
\begin{tcolorbox}[notitle,boxrule=0pt,colback=gray!20]
(dokumentiert hier Eure Benutzerkennung sowei die URL Euros Repositories bei GitHub und macht einen screenshot Eurer Github-Webseite)
\\
\href{https://github.com/FabioPlunser/FSST4BHEL}{https://github.com/FabioPlunser/FSST4BHEL}
\end{tcolorbox}
\begin{figure}[!h]
  \centering
  \includegraphics[scale=0.6]{Bilder/Gtihub-screen}
  \caption{Github Seite}
\end{figure}



%#-----------------------------------------------------------------------------------------------------------------------------------------------------
%#-----------------------------------------------------------------------------------------------------------------------------------------------------
%#-----------------------------------------------------------------------------------------------------------------------------------------------------
%#-----------------------------------------------------------------------------------------------------------------------------------------------------
%#-----------------------------------------------------------------------------------------------------------------------------------------------------
\newpage
\section{Erzeugt mit clone eine Kopie Eures GitHub Repos in Eurer PyCharm Umgebung.}
\subsection{Erstellt in der Umgebung im Projekt weitere Dateien wie z.B. ein perosnalisiertes HelloWorld.py.}
\subsection{Commitet Euro Änderung in Euer lokales Repo}
\begin{tcolorbox}[notitle,boxrule=0pt,colback=gray!20]
Vorsicht: Beachtet dabei Eure Privacy  - ihr werdet das Projekt ja bei GitHub publizieren.\\

In meinem Fall wird Visual Studio Code verwendet mit der GIT-History extension.\\
Man sieht zwar nicht direkt die DateiSturktur es reicht aber vollkommen aus, auch wenn man die Commands 
weiß benötigt man nur die weboberfläche.
\end{tcolorbox}
\begin{figure}[!h]
  \centering
  \includegraphics[scale=0.35]{Bilder/meigit-lokal}
  \caption{MeinGit-History}
\end{figure}



%#-----------------------------------------------------------------------------------------------------------------------------------------------------
%#-----------------------------------------------------------------------------------------------------------------------------------------------------
%#-----------------------------------------------------------------------------------------------------------------------------------------------------
%#-----------------------------------------------------------------------------------------------------------------------------------------------------
%#-----------------------------------------------------------------------------------------------------------------------------------------------------
\section{Pusht Eure Projektänderungen nach GitHub}
(screenshots der Webansicht Eures Projekts)
\begin{figure}[!h]
  \centering
  \includegraphics[scale=0.4]{Bilder/git-push}
  \caption{History änderung nach Git push}
\end{figure}
\begin{figure}[!h]
  \centering
  \includegraphics[scale=0.6]{Bilder/github-push}
  \caption{Github Seite nach Push}
\end{figure}





\newpage
%#-----------------------------------------------------------------------------------------------------------------------------------------------------
%#-----------------------------------------------------------------------------------------------------------------------------------------------------
%#-----------------------------------------------------------------------------------------------------------------------------------------------------
%#-----------------------------------------------------------------------------------------------------------------------------------------------------
%#-----------------------------------------------------------------------------------------------------------------------------------------------------
\section{Findet einen Partner in der Klasse und berechtigt Euch wechselseitig für Eure Git-Hub-Projekte.}
\begin{tcolorbox}[notitle,boxrule=0pt,colback=gray!20]
Partner Mitterhuber Lorenz\\
Habe noch nicht ganz verstanden wich ich einstellen kann was der Lorenz machen darf und was nicht bzw. 
habe es nirgends gefunden
\end{tcolorbox}
\begin{figure}[!h]
  \centering
  \includegraphics[scale=0.4]{Bilder/github-berechtigung}
  \caption{Github-Berechtigung}
\end{figure}
\begin{figure}[!h]
  \centering
  \includegraphics[scale=0.4]{Bilder/lorenz-git}
  \caption{Git-Page vom Lorenz}
\end{figure}



\newpage
%#-----------------------------------------------------------------------------------------------------------------------------------------------------
%#-----------------------------------------------------------------------------------------------------------------------------------------------------
%#-----------------------------------------------------------------------------------------------------------------------------------------------------
%#-----------------------------------------------------------------------------------------------------------------------------------------------------
%#-----------------------------------------------------------------------------------------------------------------------------------------------------
\section{Klont das Repo Eures Partners in ein neues lokales PyChram-Projekt auf Eurem PC. Bringt dort zuerst lokal und dann auch im Remote-Repo
bei GitHub Eure Änderungen ein.}
\begin{figure}[!h]
  \centering
  \includegraphics[scale=0.6]{Bilder/copy-git-lorenz}
  \caption{Kopie von Repo von Lorenz}
\end{figure}
\begin{figure}[!h]
  \centering
  \includegraphics[scale=0.6]{Bilder/git-lorenz-push}
  \caption{Meine Änderungen}
\end{figure}
\begin{figure}[!h]
  \centering
  \includegraphics[scale=0.6]{Bilder/github-lorenz-push}
  \caption{Änderung auf Github vom Lorenz}
\end{figure}
\begin{tcolorbox}[notitle,boxrule=0pt,colback=gray!20]
In der File History kann genau nachgeschaut werden, wer, wann was geändert hat
\end{tcolorbox}









\newpage
%#-----------------------------------------------------------------------------------------------------------------------------------------------------
%#-----------------------------------------------------------------------------------------------------------------------------------------------------
%#-----------------------------------------------------------------------------------------------------------------------------------------------------
%#-----------------------------------------------------------------------------------------------------------------------------------------------------
%#-----------------------------------------------------------------------------------------------------------------------------------------------------
\section{Was ist unklar geblieben, wo braucht ihr Hilfe?}
Was meinen sie mit der Benutzerkennung?\\ 
Wie kann man die Berechtigungen genauer bestimmen habe das nicht rausgefunden.
\end{document}

