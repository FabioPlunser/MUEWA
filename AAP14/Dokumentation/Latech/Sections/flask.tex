\graphicspath{{Bilder}}
\section{Flask}
\begin{tcolorbox}[notitle,boxrule=0pt,colback=gray!20]
Einmal eine Flask app zu programmieren sollte jeder tun der vorhat größere Projekte in Python zu programmieren.\\\\
Da die Originale Dokumentation teilweise unverständlich ist und es war schwierig zum ersten mal mit dem $@$ Decorators zu arbeiten.\\
Daher habe ich ein wirklich sehr gut erklärtes Tutorial vom YT-Kanal 
\href{https://www.youtube.com/watch?v=MwZwr5Tvyxo&list=PL-osiE80TeTs4UjLw5MM6OjgkjFeUxCYH}{Corey Schafer}durchgearbeitet.
\end{tcolorbox}

\begin{tcolorbox}[notitle,boxrule=0pt,colback=gray!20]
Ich habe nicht jeden Meilenstein dokumentiert und da ich seit beendigung des Flask Projektes schon  für die Diplomarbeit mir PyQt5 angelernt habe,
bezieht sich die Dokumentation auf das "fertige" Projekt. Man kann es jedoch nicht fertig nennen da so eine Webseite quasi nie fertig ist. 
\end{tcolorbox}

\begin{tcolorbox}[notitle,boxrule=0pt,colback=gray!20]
Es wurde eine Blog Webseite Programmiert mit Login und Register, wo auch die Daten einer Datenbank Datei gespeichert werden. 
\end{tcolorbox}

\subsection{Aha Momente}
\begin{tcolorbox}[notitle,boxrule=0pt,colback=gray!20]
Die Größten aha Momente waren wo ich gelernt habe wie man Bootstrap einbindet und was Bootstrap ist, \\
Wie man Python Projekte mit $\_init\_.py$ files packages macht anstatt modules und damit auch absolute imports.\\\\
Jedoch konnte ich diese Python Package Methode nicht auf unser Diplomarbeit GUI Python Projekt anwenden da dann die Imports, obwohl 
der python Intellisens (autocompletion) die Imports alle findet, nicht vom Interpreter gefunden werden.
\end{tcolorbox}

\subsection{Das Programm}
\begin{tcolorbox}[notitle,boxrule=0pt,colback=gray!20]
Das Vollständige Programm kann unter meinem FSST \href{https://github.com/FabioPlunser/MUEWA/tree/master/AAP14/Programms/Flask}{Github} \\
Gefunden werden aber noch nicht gewonloadet da es noch komplett Privat ist
\end{tcolorbox}
        
    
    

